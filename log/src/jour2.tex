The Lagrangian for the model is given by:
\begin{equation}
  \mathcal{L} = -\frac{1}{4}F_{\mu\nu}F^{\mu\nu} +\left(D_\mu\phi\right)^*\left(D^\mu\phi\right)-\frac{\lambda}{2}\left(\phi^*\phi-v^2\right)^2
\end{equation}
where
\begin{align}
  &F_{\mu\nu} = \partial_\mu\phi_\nu-\partial_\nu\phi_\mu\\
  &D_\mu\phi = \partial_\mu \phi - ieA_\mu\phi
\end{align}

The equations of motion obtained by varying the action with respect to
$\phi,\ \phi^*$ and $A_\mu$ are:
\begin{align}
  & \partial^\nu F_{\nu\mu} = ie\left(\phi^*D_{\mu}-\phi\left(D_\mu\phi\right)^*\right)\label{eq:motamu}\\
  & \left(D^\mu D_\mu \phi\right)^*+\lambda\left(\phi^*\phi-v^2\right)\phi^* = 0\label{eq:phistar}\\
  & D^\mu D_\mu \phi+\lambda\left(\phi^*\phi-v^2\right)\phi = 0\label{eq:phi}
\end{align}
We assume a static solution, with a polar symmetry. We choose the
fields in the form:
\begin{align}
  &A_i = -\frac{1}{er}\varepsilon_{ij}n_j A(r)\\
  &\phi = ve^{i\theta}F(r)
\end{align}
By substituing this ansatz in the equations of motion, we find two new
equations for $A(r)$ and $F(r)$. For the left hand side of (\ref{eq:motamu})
we have
\begin{align}
  &\partial_iA_j = \frac{1}{e}\left[\left(\frac{2A}{r^2}-\frac{A'}{r}\right)\varepsilon_{jk}n_in_k+\frac{A}{r^2}\varepsilon_{ij}\right]\\
  \Longrightarrow\quad& \partial_iA_j-\partial_jA_i = \frac{1}{e}\left[\left(\frac{2A}{r^2}-\frac{A'}{r}\right)\left(\varepsilon_{jk}n_i-\varepsilon_{ik}n_j\right)n_k+\frac{2A}{r^2}\varepsilon_{ij}\right]\\
  \Longrightarrow\quad& \partial^iF_{ij} = -\frac{1}{e}\left[\frac{A'}{r}\right]' \varepsilon_{jk}n_k
\end{align}
and for the right hand side:
\begin{align}
  &D_i \phi = v^2F\left(i\partial_i\theta F+F'n_i+i\frac{AF}{r}\varepsilon_{ij}n_j\right)\\
  \Longrightarrow\quad&ie\left[\phi^*D_i\phi-\phi D_i\phi^*\right] = ie2i\Im(\phi^*D_i\phi)\\
  &\hphantom{ie\left[\phi^*D_i\phi-\phi D_i\phi^*\right]} = -2ev^2F\left(\partial_i\theta F+\frac{AF}{r}\varepsilon_{ij}n_j\right)
\end{align}
Recalling that
\begin{equation}
  \partial_i\theta = -\frac{1}{r}\varepsilon_{ij}n_j
\end{equation}
we get:
\begin{equation}
  ie\left[\phi^*D_j\phi-\phi D_j\phi^*\right] = 2ev^2\frac{F^2(1-A)}{r}\varepsilon_{jk}n_k
\end{equation}
and the first equation of motion for $A$ and $F$ writes:
\begin{equation}
  -\left[\frac{A'}{r}\right]' - 2e^2v^2\frac{F^2(1-A)}{r} = 0
\end{equation}
Let's now find the second equation:
\begin{align}
  &D_i\phi = iv\frac{(A-1)F}{r}e^{i\theta}\varepsilon_{ij}n_j+vF'e^{i\theta}n_i\\
  &D_iD^i\phi = v\frac{(A-1)^2F}{r^2}e^{i\theta}-v\frac{1}{r}\frac{\mathrm d}{\mathrm d r}\left[r\frac{\mathrm d F}{\mathrm d r}\right]e^{i\theta}
\end{align}
The potential part of the equation gives:
\begin{equation}
  \lambda\left(\phi^*\phi-v^2\right)\phi = \lambda v^3(F^2-1)Fe^{i\theta}
\end{equation}
One finally obtain
\begin{equation}
  -\frac{\mathrm d}{\mathrm d r}\left[r\frac{\mathrm d F}{\mathrm dr}\right]+\lambda v^2 r (F^2-1)F+ \frac{F}{r}(A-1)^2 = 0
\end{equation}

\begin{subsection}{Numerical integration of the equations of motion}\label{subsec:integration}
  In the last paragraph, we have obtained a differential system of two
  equations:
  \begin{align}
    &-\frac{\mathrm d}{\mathrm d r}\left[r\frac{\mathrm d F}{\mathrm dr}\right]+\lambda v^2 r (F^2-1)F+ \frac{F}{r}(A-1)^2 = 0\\
    &-\frac{\mathrm d}{\mathrm d r}\left[\frac{1}{r}\frac{\mathrm d A}{\mathrm d r}\right] - 2e^2v^2\frac{F^2(1-A)}{r} = 0
  \end{align}
  with the following asymptotic behaviour:
  \begin{align}
    &F(r) \to 1,\qquad A(r) \to 1,\qquad \text{as }r\to \infty\\
    &F(r) \to 0,\qquad A(r) \to 0,\qquad \text{as }r\to 0
  \end{align}

  We are looking for a numerical scheme which converge to the analytic
  solution. Let's define for convenience the two auxiliary functions:
  \begin{align}
    M(r,x,y) = \alpha r^2(y^2-1)y\\
    N(r,x,y) = \beta ry^2(1-x)
  \end{align}
  such that the system to solve writes:
  \begin{align}
    \left\{\begin{aligned}
      r^2F''+rF'-M(r,A,F) = 0\\
      rA''-A'+N(r,A,F) = 0
    \end{aligned}\right.\label{eq:nonlinsyst}
  \end{align}
  The boundary condition at infinity is hard to implement, so in a
  first approximation, we choose a maximal radius $R$ where both
  function $A$ and $F$ eventually reach 1. We hope that the result of
  the simulation will be a good approximation of the solution if $R$
  is sufficiently large.

  We discretize the intervalle $[0,R]$ into $N$ points in the
  following way:
  \begin{equation}
    r_i = ih,\quad h = \frac{R}{N+1} \quad\Longrightarrow\quad r_0 = 0,\ r_{N+1} = R
  \end{equation}
  and we write
  \begin{equation}
    a_i = A(r_i),\quad f_i = F(r_i)
  \end{equation}
  The non-linear system (\ref{eq:nonlinsyst}) gives the following
  finite differences scheme:
  \begin{align}
    \frac{r_i^2}{h^2}\left(f_{i+1}-2f_i+f_{i-1}\right)+\frac{r_i}{2h}\left(f_{i+1}-f_{i-1}\right) - M(r_i, a_i, f_i) = 0\\
    \frac{r_i}{h^2}\left(a_{i+1}-2a_i+a_{i-1}\right)-\frac{1}{2h}\left(a_{i+1}-a_{i-1}\right) + N(r_i, a_i, f_i) = 0
  \end{align}
  $\forall\ i = 1,\dots,N$ with the boundary conditions:
  \begin{equation}
    a_0 = f_0 = 0,\quad a_{N+1} = f_{N+1} = 1
  \end{equation}

  \begin{figure}[h!]
    \input{../figures/a_and_f_funcs_vortex.tex}
    \caption{\em Numerical integration of the non-linear system of
      equation for the vortex. We have chosen $R = 7$, $N = 400$,
      $\alpha = \beta = 1$. We note that obviously, for $r\to 0$, we
      have $f(r)\neq O(r^2)$.}
    \label{fig:a_and_f_funcs}
  \end{figure}

  In order to solve for the unknown $a_i$, $f_i$, let's write
  \begin{equation}
    x_i =\left\{
    \begin{aligned}
      &a_i,\quad\forall i = 1,\dots,N\\
      &f_i,\quad\forall i = N+1,\dots,2N\\
    \end{aligned}\right.
  \end{equation}
  and the numerical scheme becomes
  \begin{equation}
    \vec g(x_1,\dots,x_{2N}) =
    \begin{bmatrix}
      g_1(x_1, \dots, x_{2N})\\
      \vdots\\
      g_{2N}(x_1, \dots, x_{2N})\\
    \end{bmatrix}
    =
    \begin{bmatrix}
      0\\
      \vdots\\
      0
    \end{bmatrix}
  \end{equation}
  where
  \begin{align}
    g_1(x) &= \frac{r_1}{h^2}\left(x_2-2x_1\right)-\frac{1}{2h}x_2+N(r_1,x_1,x_{N+1})\\
    g_N(x) &= \frac{r_N}{h^2}\left(1-2x_N+x_{N-1}\right)-\frac{1}{2h}\left(1-x_{N-1}\right)+N(r_N,x_N,x_{2N})\\
    g_{N+1}(x)& = \frac{r_1^2}{h^2}\left(x_{N+2}-2x_{N+1}\right)+\frac{r_1}{2h}x_{N+2}-M(r_1,x_1, x_{N+1})\\
    g_{2N}(x) &= \frac{r_N^2}{h^2}\left(1-2x_{2N}+x_{2N-1}\right)+\frac{r_N}{2h}\left(1-x_{2N-1}\right)-M(r_N,x_N, x_{2N})
  \end{align}
  \begin{multline}
    g_i(x) = \frac{r_i}{h^2}\left(x_{i+1}-2x_i+x_{i-1}\right)-\frac{1}{2h}\left(x_{i+1}-x_{i-1}\right)\\
    +N(r_i, x_i, x_{N+i}),\quad
    \forall i = 2,\dots,N-1\\
  \end{multline}
  \begin{multline}
    g_i(x) = \frac{r_{i-N}^2}{h^2}\left(x_{i+1}-2x_i+x_{i-1}\right)-\frac{r_{i-N}}{2h}\left(x_{i+1}-x_{i-1}\right)\\
    +M(r_{i-N}, x_{i-N}, x_i),\quad
    \forall i = N+2,\dots,2N-1\\
  \end{multline}
  We then use Newton-Raphson method to find the solution for
  $x_i$'s:
  \begin{equation}
    x^{i+1} = x^i - \frac{\vec g(x)}{J[\vec g]_x}\quad \Longleftrightarrow\quad J[\vec g]_x^{-1}\left(x^{i+1}-x^i\right) = \vec g(x)
  \end{equation}
  where $J[\vec g]_x$ is the Jacobian matrix of $\vec g$ evaluated at
  $x$. The solution is given by $x = \lim_{i\to\infty}x^i$, and the
  convergence is faster if we choose cleverly the initial point $x^0$.

  The results are shown on Fig. \ref{fig:a_and_f_funcs}.

\end{subsection}
