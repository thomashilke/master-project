The Lagrangian we are interested in for a first try is
\begin{equation}
  \mathcal{L} = \left(\phi^*\phi+C\right)R+(D_\mu\phi)^*(D^\mu\phi)-\frac{\lambda}{2}(\phi^*\phi-v^2)^2 -\frac{1}{4}F_{\mu\nu}F^{\mu\nu}\label{eq:gravitylagrangian}
\end{equation}
where $R$ is the Ricci scalar, $C,\ \lambda,\ v$ real constants, $\phi$ is a
complex scalar field, and
\begin{align}
  D_\mu\phi = (\partial_\mu-ieA_\mu)\phi\\
  F_{\mu\nu} = \nabla_\mu A_\nu-\nabla_\nu A_\mu
\end{align}
with $\nabla_\mu A_\nu = \partial_\mu A_{\nu}-\Gamma_{\mu\nu}^\alpha
A_{\alpha}$ the usual covariant derivative under change of coordinate
system. It is understood that the indices are moved up and down using the metric tensor $g^{\mu\nu}$ or its inverse: $A^\mu = g^{\mu\alpha}A_\alpha$.

The lagrangian (\ref{eq:gravitylagrangian}) is invariant under $U(1)$
gauge transformation and under arbitrary change of coordinate.

\begin{subsection}{Field equations}
  Let's split the Lagrangian into two parts, as usually done:
  \begin{align}
    \mathcal{L} = \mathcal{L}_\text{G}+\mathcal{L}_\text{M}
    \end{align}
  where
  \begin{align}
    \mathcal{L}_\text{G} &= \left(\phi^*\phi+C\right)R,\\
    \mathcal{L}_\text{M} &= (D_\mu\phi)^*(D^\mu\phi)-\frac{\lambda}{2}(\phi^*\phi-v^2)^2 -\frac{1}{4}F_{\mu\nu}F^{\mu\nu}  
  \end{align}
  
  The variation of the action for the gravitational field with respect
  to the metric gives:
  \begin{align}
    \delta S_\text{G} &= \int\mathrm{d}^4x\delta\left[\sqrt{-g}(\phi^*\phi+C)R\right]\\
    &= \int\mathrm{d}^4x\left[\delta\sqrt{-g}(\phi^*\phi+C) R + \sqrt{-g}(\phi^*\phi+C)\delta R\right]
  \end{align}
  By definition of the Ricci's scalar, we have:
  \begin{align}
    \delta R = \delta\left(g^{\mu\nu}R_{\mu\nu}\right) = \delta g^{\mu\nu}R_{\mu\nu}+g^{\mu\nu}\delta R_{\mu\nu}
  \end{align}
  The first term lead to the usual Einstein tensor, and the second
  term can be written as a total covariant derivative:
  \begin{align}
    \delta R_{\mu\nu} &= \delta \left(\partial_{\alpha}\Gamma_{\mu\nu}^{\alpha}-\partial_{\nu}\Gamma_{\alpha\mu}^{\alpha}+\Gamma_{\alpha\rho}^{\alpha}\Gamma_{\mu\nu}^{\rho}-\Gamma_{\nu\rho}^{\alpha}\Gamma_{\alpha\mu}^{\rho}\right)\\
    &= \partial_{\alpha}\delta \Gamma_{\mu\nu}^{\alpha} -\Gamma_{\alpha\mu}^{\rho}\delta\Gamma_{\nu\rho}^{\alpha}+\Gamma_{\alpha\rho}^{\alpha}\delta\Gamma_{\mu\nu}^{\rho} - \left(\partial_{\nu}\delta \Gamma_{\alpha\mu}^{\alpha}-\Gamma_{\mu\nu}^{\rho}\delta\Gamma_{\alpha\rho}^{\alpha}+\Gamma_{\nu\rho}^{\alpha}\delta\Gamma_{\alpha\mu}^{\rho}\right)\\
    &=\nabla_\alpha\delta\Gamma_{\mu\nu}^{\alpha}-\nabla_\nu\delta\Gamma_{\alpha\mu}^{\alpha}
  \end{align}
  In the last equality, we add two terms which cancel each other, in
  order to rebuild a valid covariant derivative of the variation of
  the Christoffel symbols.

  By definition of the covariant derivative we have:
  \begin{align}
    g^{\mu\nu}\left(\nabla_\alpha\delta\Gamma_{\mu\nu}^{\alpha}-\nabla_\nu\delta\Gamma_{\mu\nu}^{\alpha}\right) &= \nabla_\alpha\left(g^{\mu\nu}\delta\Gamma_{\mu\nu}^{\alpha}\right)-\nabla_\nu\left(g^{\mu\nu}\delta\Gamma_{\alpha\mu}^{\alpha}\right)\\
    & = \nabla_\alpha\left(g^{\mu\nu}\delta\Gamma_{\mu\nu}^{\alpha}-g^{\mu\alpha}\delta\Gamma_{\beta\mu}^{\beta}\right)\\
    & = \nabla_\alpha W^\alpha
  \end{align}
  and using the fact that, for an arbitrary vector:
  \begin{align}
    \partial_\alpha\left(\sqrt{-g}A^\alpha\right) = \sqrt{-g}\nabla_\alpha A^{\alpha}
  \end{align}
  the variation of the action gives:
  \begin{align}
    \delta S = \int\mathrm{d}^4\bigg[&\sqrt{-g}\left(\phi^*\phi+C\right)\left(R_{\mu\nu}-\frac{1}{2}g_{\mu\nu}R\right)\delta g^{\mu\nu}\\
    &+ (\phi^*\phi+C)\partial_{\alpha}\left(\sqrt{-g}W^\alpha\right)\bigg] \\
  \end{align}
  At this stage we may integrate by part the last term, drop the
  boundary terms,and express $W^\alpha$ in terms of the variation of
  the metric. We eventually get:
  \begin{align}
    \delta S = \int\mathrm{d}^4x\sqrt{-g}\left((\phi^*\phi+C)G_{\mu\nu}-\partial_\alpha\phi^2 W^{\alpha}\right)
  \end{align}
  \begin{align}
    W^\alpha = &g^{\mu\nu}\Gamma_{\alpha\mu\nu}\delta g^{\sigma \alpha}-g^{\mu\sigma} \Gamma_{\alpha\mu\nu}\delta g^{\nu \alpha}\\
    &+ \frac{1}{2}\left(g^{\mu\nu}g^{\sigma\rho}- g^{\mu\sigma}g^{\nu\rho}\right)\left(\delta g_{\rho \mu,\nu}+\delta g_{\rho \nu,\mu}-\delta g_{\mu\nu,\rho}\right)\\
    =& g^{\mu\nu}\Gamma_{\alpha\mu\nu}\delta g^{\sigma \alpha}-g^{\mu\sigma} \Gamma_{\alpha\mu\nu}\delta g^{\nu \alpha}\\
    &+ \left(g^{\mu\nu}g^{\rho\sigma}-g^{\mu\rho}g^{\nu\sigma}\right)\delta g_{\rho\mu,\nu}
  \end{align}
  Substituting this result into the variation of the action and
  integrating by part the last term leads to:
  \begin{align}
    \delta S_{\text{G}} =& \int\mathrm d^4x\sqrt{-g}G_{\mu\nu}\delta g^{\mu\nu}\\
    &+\int\mathrm d^4x\sqrt{-g} \partial_\sigma\phi^2\left(g^{\mu\sigma} \Gamma_{\alpha\mu\nu}\delta g^{\nu \alpha}-g^{\mu\nu}\Gamma_{\alpha\mu\nu}\delta g^{\sigma \alpha}\right)\\
    &+\int\mathrm d^4x\sqrt{-g} \partial_\nu\left[\sqrt{-g}\partial_\sigma\phi^2\left(g^{\mu\nu}g^{\rho\sigma}-g^{\mu\rho}g^{\nu\sigma}\right)\right]\delta g_{\rho\mu}
  \end{align}
\end{subsection}
Using the fact that:
\begin{align}
  &\partial_\mu\left(\sqrt{-g}A_{\nu}\right) = \sqrt{-g}\left(\nabla_\mu A_\nu +\Gamma_{\mu\nu}^{\lambda}A_{\lambda}+\Gamma_{\mu\lambda}^{\lambda}A_{\nu}\right)\\
  &\partial_\mu\left(\sqrt{-g}A^{\nu}\right) = \sqrt{-g}\left(\nabla_\mu A^\nu -\Gamma_{\mu\lambda}^{\nu}A^{\lambda}+\Gamma_{\mu\lambda}^{\lambda}A_{\nu}\right)
\end{align}
the last term writes:
\begin{align}
  & \sqrt{-g}\nabla_{\nu}\partial_\sigma\phi^2\left(g^{\mu\nu}g^{\rho\sigma}-g^{\mu\rho}g^{\nu\sigma}\right)\delta g_{\rho\mu}\\
  +& \sqrt{-g}\left(\Gamma_{\nu\sigma}^{\lambda}\partial_\lambda\phi^2+\Gamma^{\lambda}_{\nu\lambda}\partial_\sigma\phi^2\right)\left(g^{\mu\nu}g^{\rho\sigma}-g^{\mu\rho}g^{\nu\sigma}\right)\delta g_{\rho\mu}\\
  +& \sqrt{-g}\partial_\sigma\phi^2\partial_\nu\left(g^{\mu\nu}g^{\rho\sigma}-g^{\mu\rho}g^{\nu\sigma}\right)\delta g_{\rho\mu}
\end{align}
Expressing these terms with the variation of the inverse metric
$\delta g_{\rho\mu} = -g_{\rho\beta}g_{\mu\delta}\delta g^{\beta\delta}$ we get:
\begin{align}
  &\sqrt{-g}\left(\nabla_\nu\partial^\nu\phi^2g_{\beta\delta}-\nabla_\delta\partial_\beta\phi^2\right)\delta g^{\beta\delta}\\
  &+\sqrt{-g}\left(\partial_\lambda \phi^2g^{\lambda\nu}\Gamma_{\delta\beta\nu}-\partial_\beta\phi^2 g^{\lambda\nu}\Gamma_{\delta\lambda\nu}\right)\delta g^{\beta\delta}
\end{align}
and the variation of the gravity part of the lagrangian reduce to:
\begin{align}
  \delta \mathcal S_{\text{G}} = \int\mathrm d^4x\sqrt{-g}\left[G_{\mu\nu}+\nabla^2\phi^2g_{\mu\nu}-\nabla_\mu\nabla_\nu\phi^2\right]\delta g^{\mu\nu}
\end{align}
Noting that, due to the symmetry of the lower indices of the
Christoffel symbols, the energy-strength tensor do not depend on the
metric, neither does the covariant derivative of the Higgs field
$\phi$, the variation of the matter's action with respect to the
metric is easy to compute, and gives:
\begin{align}
  \delta\mathcal{S}_\phi = \int\mathrm d^4x\sqrt{-g}\left(\left(\mathrm{D}_\mu\phi\right)^*\mathrm{D}_\nu\phi+\frac{1}{2}F_{\mu\alpha}F^\alpha_{\hphantom{\alpha}\nu}\right)\delta g^{\mu\nu}
\end{align}
The field equation for the gravity eventually writes:
\begin{align}
  (\phi^*\phi+C)G_{\mu\nu}+\nabla^2\phi^2g_{\mu\nu}-\nabla_\mu\nabla_\nu\phi^2 + \left(\mathrm{D}_\mu\phi\right)^*\mathrm{D}_\nu\phi+\frac{1}{2}F_{\mu\alpha}F^\alpha_{\hphantom{\alpha}\nu} = 0
\end{align}
The equations for the Higgs field are:
\begin{align}
  &-R\phi^*+\mathrm{D}^\mu\mathrm{D}_\mu\phi^*+\lambda\left(\phi^*\phi -v^2\right)\phi^* = 0\\
  &\text{c. c.} = 0
\end{align}
where
\begin{align}
  \mathrm{D}^\mu\mathrm{D}_\mu\phi^* = \nabla^\mu(\mathrm{D}_\mu\phi^*)+ieA^\mu\left(\mathrm{D}_\mu\phi^*\right)
\end{align}
Eventually, the field equation for the gauge field is:
\begin{align}
  \nabla_\nu F^{\mu\nu} = ie\left(\phi^*\mathrm{D}^\mu\phi-\phi\mathrm{D}^\mu\phi^*\right)
\end{align}
Hence, the system of field equation to solve is the following:
\begin{align}
  &(\phi^*\phi+C)G_{\mu\nu}+\nabla^2\phi^2g_{\mu\nu}-\nabla_\mu\nabla_\nu\phi^2 + \left(\mathrm{D}_\mu\phi\right)^*\mathrm{D}_\nu\phi+\frac{1}{2}F_{\mu\alpha}F^\alpha_{\hphantom{\alpha}\nu} = 0\label{eq:gravfieldeq_metric}\\
  &\nabla_\nu F^{\mu\nu} + ie\left(\phi\mathrm{D}^\mu\phi^*-\phi^*\mathrm{D}^\mu\phi\right) = 0\label{eq:gravfieldeq_gauge}\\
  &-R\phi^*+\mathrm{D}^\mu\mathrm{D}_\mu\phi^*+\lambda\left(\phi^*\phi -v^2\right)\phi^* = 0\label{eq:gravfieldeq_higgs}\\
  &\text{c. c.} = 0
\end{align}
where $G_{\mu\nu}$ is Einstein's tensor.
\begin{subsection}{Spherically symmetric ansatz}
  The field equations are dimension-agnostic. We choose to work in a
  $(2+1)$ dimensions space, with a spherical symmetry about the
  coordinate origin. We assume that the most general form of a
  spherically symmetric metric tensor is the following:
  \begin{align}
    \mathrm{d}s^2 = -\beta(r)\mathrm{d}t^2+\alpha(r)\mathrm{d}r^2 + r^2\mathrm{d}\theta^2
  \end{align}
  The ansatz for the Higgs and gauge fields is:
  \begin{align}
    &\phi(r,\theta) = ve^{i\theta}F(r)\\
    &A_i(r,\theta) = -\frac{1}{er}\varepsilon_{ij}n_jK(r)
  \end{align}
 The only non zero componants of Einstein's tensor are:
  \begin{align}
    &G_{11} = -\frac { \alpha' \beta }{ 2r \alpha^{2}r}\\
    &G_{22} =   -1/2\frac {\beta' }{r\beta}\\
    &G_{33} =   -\frac{{r}^{2}}{4} \frac{ 2\alpha\beta\beta'' - \alpha\beta'^2 - \alpha'\beta'\beta}{ \alpha^2 \beta^2}\\
  \end{align}
  With this ansatz the field equation reduce to five non trivial
  relation. The diagonal componants of (\ref{eq:gravfieldeq_metric})
  give three equations:
  \begin{align}
\frac{2 v^2 \beta  F'^2}{\alpha }-\frac{c \beta  \alpha '}{2 r \alpha ^2}-\frac{v^2 F^2 \beta  \alpha '}{2 r \alpha ^2}+v^2 F F' \left(\frac{2 \beta }{r \alpha }-\frac{\beta  \alpha '}{\alpha ^2}\right)+\frac{2 v^2 F \beta  F''}{\alpha }=0\\
    v^2 F'^2-\frac{K'^2}{2 e^2 r^2}-\frac{c \beta '}{2 r \beta }-\frac{v^2 F^2 \beta '}{2 r \beta }+v^2 F F' \left(-\frac{2}{r}-\frac{\beta '}{\beta }\right) = 0
  \end{align}
  \begin{multline}
    -\frac{2 r^2 v^2 F'^2}{\alpha }-\frac{K'^2}{2 e^2 \alpha }+v^2 F F' \left(\frac{r^2 \alpha '}{\alpha ^2}-\frac{r^2 \beta '}{\alpha  \beta }\right)-\frac{2 r^2 v^2 F F''}{\alpha }\\
        +(c+v^2F^2) \left(\frac{r^2 \alpha ' \beta '}{4 \alpha ^2 \beta }+\frac{r^2 \beta '^2}{4 \alpha  \beta ^2}-\frac{r^2 \beta ''}{2 \alpha  \beta }\right)+v^2 F^2 (1-K)^2 = 0
  \end{multline}
  Equation (\ref{eq:gravfieldeq_gauge}) reduce to the $\theta,\theta$
  componant:
  \begin{align}
    -2 e v^2 F^2 (K-1)+\frac{1}{e \alpha }\left(\frac{K' \alpha '}{2  \alpha }+\frac{K'}{r}-\frac{K' \beta '}{2 \beta }-K''\right)=0
  \end{align}
  And equation (\ref{eq:gravfieldeq_higgs}) gives:
  \begin{multline}
    \frac{v F}{\alpha} \left(-\frac{\alpha '}{r \alpha }+\frac{\beta '}{r  \beta }-\frac{\alpha ' \beta '}{2 \alpha  \beta }-\frac{\beta '^2}{2  \beta ^2}+\frac{\beta ''}{  \beta }\right)-\frac{v F \left(K-1\right)^2}{r^2}\\
    + v F' \left(\frac{1}{r \alpha }-\frac{\alpha '}{2 \alpha ^2}+\frac{\beta '}{2 \alpha  \beta }\right)+\frac{v F''}{\alpha }+v^3 \lambda  F \left(F^2-1\right)=0
  \end{multline}
Since we have five equations and only four degrees of freedom, for the
system to be consistant they must be dependant, and one equation
should find how one equation can be expressed in terms of the others.
\end{subsection}
