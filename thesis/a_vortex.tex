\begin{section}{Numerical integration of the equations of motion: the case of the Vortex}\label{ann:num}
  We assume that the field equations for the two degrees of freedom
  $A$ and $F$ are known, and given by
  \begin{align}
    \begin{aligned}
      &-\frac{\mathrm d}{\mathrm d r}\left[r\frac{\mathrm d F}{\mathrm dr}\right]+\lambda v^2 r (F^2-1)F+ \frac{F}{r}(A-1)^2 = 0,\\
      &-\frac{\mathrm d}{\mathrm d r}\left[\frac{1}{r}\frac{\mathrm d A}{\mathrm d r}\right] - 2e^2v^2\frac{F^2(1-A)}{r} = 0
    \end{aligned}\label{eq:fequs}
  \end{align}
  with the following asymptotic behaviour:
  \begin{align}
    &F(r) \to 1,\qquad A(r) \to 1,\qquad \text{as }r\to \infty,\\
    &F(r) \to 0,\qquad A(r) \to 0,\qquad \text{as }r\to 0.
  \end{align}
  
  We are looking for a numerical scheme which converge to a solution
  of (\ref{eq:fequs}) satisfiying these boundary conditions. Let's
  define for convenience the two auxiliary functions:
  \begin{align}
    M(r,x,y) = \alpha r^2(y^2-1)y,\\
    N(r,x,y) = \beta ry^2(1-x)
  \end{align}
  such that the system to solve reads:
  \begin{align}
    \left\{\begin{aligned}
      r^2F''+rF'-M(r,A,F) = 0,\\
      rA''-A'+N(r,A,F) = 0.
    \end{aligned}\right.\label{eq:nonlinsyst}
  \end{align}
  The boundary condition at infinity is hard to implement, so in a
  first approximation, we choose a maximal radius $R$ where both
  function $A$ and $F$ eventually reach 1. We hope that the result of
  the simulation will be a good approximation of the solution if $R$
  is sufficiently large.
  
  We discretize the interval $[0,R]$ into $N$ points uniformly distributed
  \begin{equation}
    r_i = ih,\quad h = \frac{R}{N+1} \quad\Longrightarrow\quad r_0 = 0,\ r_{N+1} = R,
  \end{equation}
  and we write
  \begin{equation}
    a_i = A(r_i),\quad f_i = F(r_i).
  \end{equation}
  The non-linear system (\ref{eq:nonlinsyst}) gives rise to the following
  finite differences scheme:
  \begin{align}
    \frac{r_i^2}{h^2}\left(f_{i+1}-2f_i+f_{i-1}\right)+\frac{r_i}{2h}\left(f_{i+1}-f_{i-1}\right) - M(r_i, a_i, f_i) = 0,\\
    \frac{r_i}{h^2}\left(a_{i+1}-2a_i+a_{i-1}\right)-\frac{1}{2h}\left(a_{i+1}-a_{i-1}\right) + N(r_i, a_i, f_i) = 0
  \end{align}
  $\forall\ i = 1,\dots,N$ with the boundary conditions:
  \begin{equation}
    a_0 = f_0 = 0,\quad a_{N+1} = f_{N+1} = 1.
  \end{equation}
%  \begin{figure}[h!]
%    \input{figures/a_and_f_funcs_vortex.tex}
%    \caption{\em Numerical integration of the non-linear system of
%      equation for the vortex. We have chosen $R = 7$, $N = 400$,
%      $\alpha = \beta = 1$. We note that obviously, for $r\to 0$, we
%      have $f(r)\neq O(r^2)$.}
%    \label{fig:a_and_f_funcs}
%  \end{figure}
  In order to solve for the unknown $a_i$, $f_i$, let's write
  \begin{equation}
    x_i =\left\{ 
    \begin{aligned}
      &a_i,\quad\forall i = 1,\dots,N\\
      &f_i,\quad\forall i = N+1,\dots,2N\\
    \end{aligned}\right.
  \end{equation}
  and the numerical scheme becomes
  \begin{equation}
    \vec g(x_1,\dots,x_{2N}) = 
    \begin{bmatrix}
      g_1(x_1, \dots, x_{2N})\\
      \vdots\\
      g_{2N}(x_1, \dots, x_{2N})\\
    \end{bmatrix}
    = 
    \begin{bmatrix}
      0\\
      \vdots\\
      0
    \end{bmatrix}
  \end{equation}
  where
  \begin{align}
    g_1(x) &= \frac{r_1}{h^2}\left(x_2-2x_1\right)-\frac{1}{2h}x_2+N(r_1,x_1,x_{N+1}),\\
    g_N(x) &= \frac{r_N}{h^2}\left(1-2x_N+x_{N-1}\right)-\frac{1}{2h}\left(1-x_{N-1}\right)+N(r_N,x_N,x_{2N}),\\
    g_{N+1}(x)& = \frac{r_1^2}{h^2}\left(x_{N+2}-2x_{N+1}\right)+\frac{r_1}{2h}x_{N+2}-M(r_1,x_1, x_{N+1}),\\
    g_{2N}(x) &= \frac{r_N^2}{h^2}\left(1-2x_{2N}+x_{2N-1}\right)+\frac{r_N}{2h}\left(1-x_{2N-1}\right)-M(r_N,x_N, x_{2N}),
  \end{align}
  \begin{multline}
    g_i(x) = \frac{r_i}{h^2}\left(x_{i+1}-2x_i+x_{i-1}\right)-\frac{1}{2h}\left(x_{i+1}-x_{i-1}\right),\\
    +N(r_i, x_i, x_{N+i}),\quad
    \forall i = 2,\dots,N-1,\\
  \end{multline}
  \begin{multline}
    g_i(x) = \frac{r_{i-N}^2}{h^2}\left(x_{i+1}-2x_i+x_{i-1}\right)-\frac{r_{i-N}}{2h}\left(x_{i+1}-x_{i-1}\right),\\
    +M(r_{i-N}, x_{i-N}, x_i),\quad
    \forall i = N+2,\dots,2N-1.\\    
  \end{multline}
  We then use Newton-Raphson iteration to find the solution for the
  $x_i$'s:
  \begin{equation}
    x^{i+1} = x^i - \frac{\vec g(x)}{J[\vec g]_x}\quad \Longleftrightarrow\quad J[\vec g]_x^{-1}\left(x^{i+1}-x^i\right) = \vec g(x)
  \end{equation}
  where $J[\vec g]_x$ is the Jacobian matrix of $\vec g$ evaluated at
  $x$. The solution is given by $x = \lim_{i\to\infty}x^i$, and the
  convergence is faster if we choose cleverly the initial point $x^0$.
  %The results are shown on Fig. \ref{fig:a_and_f_funcs}.
\end{section}
